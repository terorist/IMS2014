\documentclass[12pt,a4paper,titlepage,final]{article}
\usepackage[czech]{babel}
\usepackage[left=1.5cm,text={17cm,24cm},top=2.5cm]{geometry}
\usepackage[utf8]{inputenc}
\usepackage[T1, IL2]{fontenc}
\usepackage{listings}
\lstset{basicstyle=\small\ttfamily}

% Hack: should be placed after hypperref; otherwise, there are no links
\usepackage{multibib}   % Multiple bibliographies
\newcites{My}{Selected Author's Publications}
\usepackage[nottoc]{tocbibind} % Add bibbliographies into ToC

\usepackage{graphicx}   % Enhanced support for graphics
\usepackage{float}      % Figure placing [H]
\usepackage[list=true,listformat=simple]{subcaption} % Captions of multiple figures
 
%\usepackage{multirow}
%\usepackage{dblfloatfix}
\usepackage{booktabs}   % Better-looking tabs
%\usepackage{tabularx}   % Tables
\usepackage{colortbl}   % Cell colors


\newcommand{\myuv}[1]{\quotedblbase #1\textquotedblleft}
\usepackage{url}
\DeclareUrlCommand\url{\def\UrlLeft{<}\def\UrlRight{>} \urlstyle{tt}}
\usepackage{graphicx}
 \usepackage[usenames,x11names,table]{xcolor} % Use colors in listings and figures
\ifx\pdfoutput\undefined % not a pdflatex
\else
	\usepackage[pdftex]{hyperref}
	\hypersetup{
		unicode=true,
		colorlinks=true,
		plainpages=false,
		pdftitle={Retargetable Analysis of Machine Code},
		pdfauthor={Jakub Křoustek},
		pdfsubject={PhD Thesis}, 
		citecolor=DodgerBlue2,
	}
\fi

\begin {document}
\begin{titlepage}
\begin{center}
{\Large\textsc{Fakulta informačních technologií}}\\
{\Large \textsc{Vysoké učení technické v Brně}}\\

\begin{figure}[!h]
 	\centering
	 \includegraphics[height=7cm]{fit-zp2.pdf}
\end{figure}
\vspace{\stretch{0.382}}

{\Huge 1. Diskrétní simulátor řízený událostmi}\\
\LARGE
IMS projekt\\ 
\vspace{\stretch{0.618}}
\end{center}
{\Large
Vypracovali: Lenka Jalůvková (xjaluv02), Jiří Picek (xpicek01)} \\
{\Large Dne: \today}
\end{titlepage}

%\tableofcontents

\newpage
\section{Úvod} \label{uvod}

V této práci je řešena implementace diskrétního simulátoru pro modelování (\cite{peringer}, slajd 119) založeného na kalendáři událostí (\cite{peringer}, slajd 173). Chování tohoto simulátoru je předvedeno na dvou vybraných příkladech z democvičení předmětu IMS, vybrali jsme příklad Učebna a Kravín (\cite{demo1}, slajd 19 a 22).
Smyslem experimentu je demonstrovat, že náš simulační nástroj \texttt{libsim} odpovídá již existujícím simulátorům.

\subsection{Řešitelé a zdroje informací}

Projekt vypracoval tým ve složení Lenka Jalůvková a Jiří Picek. Využili jsme znalosti nabyté na přednáškách a democvičeních předmětu IMS. Problém jsme také prokonzultovali s doktorem Martinem Hrubým.

\subsection{Experimentální ověřování validity modelu}

Validita vybraných modelů byla ověřena za využití modelů v SIMLIBU odprezentovaných v rámci democvičení předmětu IMS (\cite{priklady}~\texttt{kravin.cpp}, \texttt{ucebna.cpp}). 

Výsledné modely byly testovány na školním serveru merlin.fit.vutbr.cz, systém CentOS a také na notebooku se systémem ArchLinux.
 
\section{Rozbor tématu a použitých metod/technologií}

Diskrétní simulátor modeluje systém jako diskrétní (nespojitou) posloupnost událostí v čase. Diskrétní simulace je tedy opakem simulace spojité, která kontinuálně zaznamenává dynamiku systému v čase. Spojitá simulace může být také označena jako simulace založená na činnostech. Čas je rozdělen na malé intervaly a stav systému je aktualizován na základě množiny činností, které se odehrávají v daném časovém intervalu. Protože diskrétní simulátor nemusí zpracovávat každý časový interval, mohou běžet mnohem rychleji než odpovídající spojité simulátory~\cite{simulace.info}. 

\subsection{Použité postupy}

Diskrétní simulátor je implementován v jazyce C++, který nám umožnil objektový vývoj. Programování nám zjednodušilo využití tříd, které k sobě svazují data a funkce nad nimi.

\subsection{Původ použitých metod/technologií}

Při tvorbě projektu nám byla inspirací knihovna SIMLIB, která umožňuje diskrétní i spojitou simulaci, dále demonstrační cvičení a přednášky předmětu IMS.

\section{Koncepce}

Hlavní komponentou diskrétního simulátoru je kalendář událostí, do kterého přidáváme nebo vyjímáme záznamy. 
Kalendář událostí je uspořádaná datová struktura uchovávající aktivační záznamy budoucích událostí. Každá naplánovaná budoucí událost \emph{next event} má v kalendáři záznam obsahující položky $[(acttime_{i}, priority_{i}, event_{i}), \dots]$. Kalendář umožňuje výběr prvního záznamu s nejnižším aktivačním časem a vkládání/rušení aktivačních záznamů.

Princip kalendáře událostí:
\begin{lstlisting}[label={lst:myListing}]
Inicializace kalendare udalosti a modelu
while(kalendar je neprazdny){
          Vyjmi prvni aktivacni zaznam (AZ) z kalendare
          if (acttime > T_END)
              break; Ukonceni cyklu
         Nastav cas na aktivacni cas acttime v AZ
         Proved popis chovani udalosti event v AZ
}
cas = T_END; konec simulace
\end{lstlisting}

\section{Architektura simulačního modelu/simulátoru}

Prvky simulátoru jsou implementovány v pěti hlavních částech (\texttt{event.h}, \texttt{gen.h}, \texttt{libsim.h}, \texttt{stat.h} a \texttt{store.h}), které tvoří prvky našeho simulátoru \texttt{libsim}. Tento simulátor je dále využíván v modelech Kravín a Učebna, které jsou v souborech \texttt{kravin.cpp} a \texttt{ucebna.cpp}. Níže jsou jednotlivé části blíže rozebrány.
 
\subsection{Kalendář událostí}

Kalendář událostí se nachází v souboru \texttt{libsim.h}. Fronta událostí pro kalendář je implementována jako multimapa z knihovny STL (Standard Template Library), která jednotlivé prvky seřadí dle hodnot klíčů, v našem případě časových záznamů. Pokud jsou klíče shodné, jejich pořadí není definováno, to však u simulace není podstatné, protože události se mají provádět ve stejný okamžik, takže nelze rozhodnout, která má přijít dříve.

Hlavní smyčka kalendáře je implementována podle algoritmu výše \ref{lst:myListing} v mírně pozměněném pořadí. Nejprve proběhne kontrola časové zarážky, dále se nastaví čas na čas aktivace aktuální události a poté se vybere událost a zavolá se její provedení. Kalendář se inicializuje pomocí funkce \texttt{init}, které se předává počáteční a koncový čas simulace. Tyto časy je možné kdykoli v simulaci přečíst z proměnných \texttt{TIME} a \texttt{TIME\_END}.

Po ukončení funkce \texttt{run} se kalendář nemaže. Pro smazání se explicitně volá funkce \texttt{freeSimulation}. Statistiky tak můžeme vypisovat ve více časových intervalech.

\subsection{Události}

Další důležitou částí je soubor \texttt{event.h} implementující události v simulaci. Události defaultně nemají žádné chování, to se musí vytvořit až v simulaci. Podporují aktivaci v aktuálním čase nebo za zadanou dobu.

\subsection{Sklad}

Sklad je implementován v souboru \texttt{store.h}, který obsahuje základní operace se skladem, a to zabrání skladu (metoda \texttt{Enter}) a jeho uvolnění (metoda \texttt{Leave}). Dále obsahuje funkce pro zjištění, zda je sklad prázdný (metoda \texttt{Empty}) či plný (metoda \texttt{Full}). 
 
\subsubsection{Zařízení}
Při použití skladu s defaultními hodnotami parametrů se chová jako zařízení. Jedná se o velikost skladu u konstruktoru a počet zabíraných nebo uvolňovaných míst u \texttt{Enter} a \texttt{Leave}.

\subsection{Generátor}

Generátory jsou implementovány v \texttt{gen.h}. Obsahuje pseudonáhodný generátor čísel z intervalu $\langle$0,1) a dále generátory s exponenciálním, s normálním a rovnoměrným rozložením.

\subsubsection{Pseudonáhodný generátor}

Pseudonáhodný generátor jsme implementovali ve funkci \texttt{Random}, nejprve jsme se inspirovali z opory lineárním kongruentním generátorem (\cite{opora} str. 18-20), avšak generoval příliš malá čísla i s různým nastavením parametrů, čímž nepříznivě ovlivnil výsledky simulací. Nakonec jsme jej implementovali za využití standardní C funkce \texttt{rand}.

\subsubsection{Generátor s exponenciálním rozložením}

Generátor s exponenciálním rozložením je implementován ve funkci \texttt{Exponential}. V tomto případě jsme se inspirovali knihovnou SIMLIB. Pro výpočet využívá výše popsaný pseudonáhodný generátor.

\subsubsection{Generátor s normálním rozložením}

Funkce \texttt{Uniform} obsahuje generátor s normálním rozložením. Funkci jsme implementovali dle principu uvedeného na fóru~\cite{forum}.

\subsubsection{Generátor s rovnoměrným rozložením}

Generátor s rovnoměrným rozložením je implementován ve funkci \texttt{Uniform}. Tuto funkci jsme implementovali na základě předchozích znalostí.

\subsection{Statistiky}

V souboru \texttt{stat.h} jsou implementovány statistiky, jejichž výstup následně můžeme vypsat po dokončení simulace. Statistické výpisy jsou prováděny pouze pro sklad. 

Sledované položky skladu jsou celkový počet požadavků o zabrání skladu, počet požadavků, které zaberou sklad bez čekání ve frontě a statistiky nad frontou u skladu. Sledují se příchozí požadavky ve frontě, dále jeji maximální, průměrná a aktuální délka a minimální, maximální a průměrný čas čekání. 

Funkce statistik se volají v metodách skladu v určitých případech. Pokud se do skladu vstoupí bez čekání ve frontě, zavolá se metoda \texttt{EnterNonqueued} ve které se zvýší počet požadavků o sklad a počet požadavků, které prošly bez fronty. Pokud se vstoupí do fronty, tak se uchová aktuální čas, kdy se vstupuje do fronty, zkontroluje se délka fronty pro uchování maximální délky fronty a zvýší se počet požadavků o sklad. V případě opouštění fronty se dopočítávají minimální, maximální čekací čas a počítá se celkový čas strávený ve frontě, který se při výpisu dále přepočítá na průměrný čas čekání ve frontě.

\subsection{Překlad}

Překlad probíhá pomocí našeho \texttt{Makefile} příkazem \texttt{make ucebna} nebo \texttt{make kravin}, případně lze přeložit oba příklady najednou pomocí \texttt{make run}. Výsledný překlad je následně umístěn ve složce \texttt{bin}.

\section{Podstata simulačních experimentů a jejich průběh}

Činnost simulátoru byla ověřena na dvou modelech z demonstračních cvičení, a to příklad Učebna a Kravín.

\subsection{Postup experimentování} 

V jednotlivých modelech jsme nejprve nastavili hodnoty parametrů podle příslušných zadání, následně jsme je změnili na základě výsledků. Nakonec jsme je nastavili stejně jako v případě modelů implementovaných pomocí knihovny SIMLIB (\cite{priklady}).

\subsection{Experimenty}

Prováděli jsme experimenty nad oběma příklady (Učebna, Kravín) převzatými z demonstračních cvičení předmětu IMS.

\subsubsection{ Kravín}

První experiment je prováděn při nastavení hodnot shodným se zadáním. V kravíně je 100 krav, 5 dojiček, 1 nakládací rampa a 2 auta. Auto může naložit 20 konvic. Experiment je prováděn v simulačním čase 0 -- 200 hodin.

\begin{figure}[!h] 
 	\centering
	 \includegraphics[]{kravy1.jpg}
\caption{Experiment s nastavením dle zadání}
\label{obr1}
\end{figure}

Z výsledku na obrázku~\ref{obr1} vidíme, že dojičky nejsou plně využity. Proto pro další experiment snížíme počet dojiček na 3, stejně jako v simulaci příkladu  SIMLIB (~\cite{priklady}).

\begin{figure}[!h] 
 	\centering
	 \includegraphics[]{kravy2.jpg}
\caption{Experiment se sníženým počtem dojiček}
\label{obr2}
\end{figure}

Z výsledku na obrázku~\ref{obr2} vidíme, že dojičky jsou lépe využity a že maximálně čekalo 7 krav nejdéle však 15 minut. Dále auto čeká na volnou rampu nejméně hodinu. Naplněné auto náklad převáží maximálně hodinu (30 - 60 minut), proto jsme usoudili, že by mohlo stačit pouze jedno auto, tudíž další experiment je pouze s jedním autem.

\begin{figure}[!h] 
 	\centering
	 \includegraphics[]{kravy3.jpg}
\caption{Experiment se sníženým počtem aut}
\label{obr3}
\end{figure}

Z výsledků na obrázku~\ref{obr3} vidíme, že auto přepravilo méně konvic než v předchozím případě, vyplatily by se minimálně dvě auta. 

\subsubsection{Učebna}

První experiment je prováděn při nastavení hodnot shodným se zadáním. Máme 10 počítačů, studenti přicházejí v intervalech daných exponenciálním rozložením se středem 10 min. Pokud je počítač volný, obsadí ho a pracují exp(100 min), jinak se 60\% okamžitě postaví do fronty, zbytek odchází. Po 30-60 minutách se však 20\% vrací. Experiment je prováděn v simulačním čase 0 -- 1000000 (stejně jako v demonstračních cvičeních)..

\begin{figure}[!h] 
 	\centering
	 \includegraphics[]{ucebna1.jpg}
\caption{Experiment s nastavením dle zadání}
\label{obr4}
\end{figure}

Z výsledku na obrázku~\ref{obr4} vidíme, že studenti na počítač průměrně čekali 27 min a nejdéle 259 min. Při zvýšení počtu počítačů na 20 se maximální délka výrazně sníží a průměrný čekací čas je také rozumnější, jak vidíme na obrázku~\ref{obr5}.

\begin{figure}[!h] 
 	\centering
	 \includegraphics[]{ucebna1.jpg}
\caption{Experiment se zvýšeným počtem počítačů na 20}
\label{obr4}
\end{figure}

Při počtu 25 počítačů se již netvoří fronty vůbec a studenti mohou pracovat bez čekání, jak vidíme na obrázku~\ref{obr5}.

\begin{figure}[!h] 
 	\centering
	 \includegraphics[]{ucebna3.jpg} 
\caption{Experiment se zvýšeným počtem počítačů na 25}
\label{obr5}
\end{figure}

\subsubsection{Závěry experimentů}

Bylo provedeno několik experimentů v různých situacích s rozličným nastavením vstupních parametrů. V průběhu experimentování s Kravínem byla odstraněna chyba v modelu, kdy auto bylo připraveno na rampě,  avšak konvice do něj nebyly nakládány.  Na základě experimentů jsme zjistili, že Kravín by fungoval i v případě 3 dojiček a 1 auta.

V případě Učebny by byla potřeba alespoň 20 počítačů, aby studenti nečekali dlouhé fronty a nejlépe 25, aby mohli pracovat ihned.

%udelat referencni SIMLIB printscreen +ucebna+teorie

\section{Shrnutí simulačních experimentů a závěr}

V rámci projektu vznikl diskrétní simulátor, který byl implementován v C++. Po spuštění vypisujeme statistiky. Implementovali jsme pouze základní z nich, které jsme shledali důležitými, případné další požadavky sledovaných hodnot se dají snadno doplnit.
Ve statistikách je mimo jiné i průměrná délka fronty, která se vždy počítá od času 0, i když čas simulace začíná v jiném čase.

Při experimentování s modelem kravín se naše výsledky oproti výsledkům modelů s knihovnou SIMLIB lišily zhruba o XXX\%. V případě příkladu učebna přibližně o XX\%, jak vidíme v tabulkách~\ref{srovnani} a ~\ref{srovnani2}.

%přepsat nějaké hodnoty do tabulky (např.: sledovaná hodnota (popis) - referenční hodnota - odchylka

\begin{table}[ht]
\caption{Náš simulátor vs. SIMLIB, Kravín se 3 dojičkami, 1 rampou, 2 auty}
\centering
\begin{tabular}{c c c c c}
\hline\hline
& Sledovaná hodnota & náš simulátor & SIMLIB & odchylka\\[0.5ex]
\hline
Rampa & min. čas&  60,453 & 38,4755&\\
 & max. čas & 238,830 & 223,847&\\
 & průměr. čas & 131,109 & 119,114&\\
Dojičky & min. čas & 0,008 & 0,027&\\
 & max. čas & 15,049 & 9,752&\\
 & průměr. čas & 4,312 & 2,878&\\[1ex]
\hline
\label{srovnani}
\end{tabular}
\end{table}

\begin{table}[ht]
\caption{Náš simulátor vs. SIMLIB, Učebna s 10 počítači, studenti exp(10 min)}
\centering
\begin{tabular}{c c c c c}
\hline\hline
& Sledovaná hodnota & náš simulátor & SIMLIB & odchylka\\[0.5ex]
\hline
Počítače & min. čas&  0,008 & 0,003&\\
 & max. čas & 259,938 & 198,401&\\
 & průměr. čas & 27,608 & 28,289&\\[1ex]
\hline
\label{srovnani2}
\end{tabular}
\end{table}

\newpage

\bibliographystyle{myplain}
\bibliography{bibliography}

\end{document}