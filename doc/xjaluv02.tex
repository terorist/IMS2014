\documentclass[12pt,a4paper,titlepage,final]{article}
\usepackage[czech]{babel}
\usepackage[left=1.5cm,text={17cm,24cm},top=2.5cm]{geometry}
\usepackage[utf8]{inputenc}
\usepackage[T1, IL2]{fontenc}
\usepackage{listings}
\lstset{basicstyle=\small\ttfamily}

% Hack: should be placed after hypperref; otherwise, there are no links
\usepackage{multibib}   % Multiple bibliographies
\newcites{My}{Selected Author's Publications}
\usepackage[nottoc]{tocbibind} % Add bibbliographies into ToC

\newcommand{\myuv}[1]{\quotedblbase #1\textquotedblleft}
\usepackage{url}
\DeclareUrlCommand\url{\def\UrlLeft{<}\def\UrlRight{>} \urlstyle{tt}}
\usepackage{graphicx}
 \usepackage[usenames,x11names,table]{xcolor} % Use colors in listings and figures
\ifx\pdfoutput\undefined % not a pdflatex
\else
	\usepackage[pdftex]{hyperref}
	\hypersetup{
		unicode=true,
		colorlinks=true,
		plainpages=false,
		pdftitle={Retargetable Analysis of Machine Code},
		pdfauthor={Jakub Křoustek},
		pdfsubject={PhD Thesis}, 
		citecolor=DodgerBlue2,
	}
\fi

\begin {document}
\begin{titlepage}
\begin{center}
{\Large\textsc{Fakulta informačních technologií}}\\
{\Large \textsc{Vysoké učení technické v Brně}}\\

\begin{figure}[!h]
 	\centering
	 \includegraphics[height=7cm]{fit-zp2.pdf}
\end{figure}
\vspace{\stretch{0.382}}

{\Huge 1. Diskrétní simulátor řízený událostmi}\\
\LARGE
IMS projekt\\ 
\vspace{\stretch{0.618}}
\end{center}
{\Large
Vypracovali: Lenka Jalůvková (xjaluv02), Jiří Picek (xpicek01)} \\
{\Large Dne: \today}
\end{titlepage}

\tableofcontents

\newpage
\section{Úvod} \label{uvod}

V této práci je řešena implementace diskrétního simulátoru pro modelování (\cite{peringer}, slajd 119) založeného na kalendáři událostí (\cite{peringer}, slajd 173). Chování tohoto simulátoru je předvedeno na dvou vybraných příkladech z democvičení předmětu IMS, vybrali jsme příklad Kravín a Vlek (\cite{demo1}, slajd 22 a 28).

\subsection{Řešitelé a zdroje informací}

Projekt vypracoval tým ve složení Lenka Jalůvková a Jiří Picek. Využili jsme znalosti nabyté na přednáškách a democvičeních předmětu IMS.

\subsection{Experimentální ověřování validity modelu}

Validita modelu byla ověřena podle modelů odprezentovaných v rámci democvičení předmětu IMS. Dále jsme validitu vybraných modelů ověřili za využití SIMLIBU (~\cite{priklady}~\texttt{kravin.cpp}, \texttt{lyzar.cpp}).
 
\section{Rozbor tématu a použitých metod/technologií}

Diskrétní simulátor modeluje systém jako diskrétní (nespojitou) posloupnost událostí v čase. Diskrétní simulace je tedy opakem simulace spojité, která kontinuálně zaznamenává dynamiku systému v čase. Spojitá simulace může být také označena jako simulace založená na činnostech. Čas je rozdělen na malé intervaly a stav systému je aktualizován na základě množiny činností, které se odehrávají v daném časovém intervalu. Protože diskrétní simulace nemusí zpracovávat každý časový interval, mohou běžet mnohem rychleji než odpovídající spojité simulace~\cite{simulace.info}. 

\subsection{Použité postupy}

Diskrétní simulátor je implementován v jazyce C++, který nám umožnil objektový vývoj.

\subsection{Původ použitých metod/technologií}
\section{Koncepce}
ABSTRAKTNI popis programu bez nazvu trid.

Hlavní komponentou diskrétního simulátoru je kalendář událostí, do kterého přidáváme nebo vyjímáme záznamy. Popsat Procesy, Zařízení, Obslužná linka??

\subsection{Kalendář událostí}

Kalendář událostí je uspořádaná datová struktura uchovávající aktivační záznamy budoucích událostí. Každá naplánovaná budoucí událost \emph{next event} má v kalendáři záznam obsahující položky $[(acttime_{i}, priority_{i}, event_{i}), \dots]$. Kalendář umožňuje výběr prvního záznamu s nejnižším aktivačním časem a vkládání/rušení aktivačních záznamů.

Princip kalendáře událostí:
\begin{lstlisting}
Inicializace kalendare udalosti a modelu
while(kalendar je neprazdny){
          Vyjmi prvni aktivacni zaznam (AZ) z kalendare
          if (acttime > T_END)
              break; Ukonceni cyklu
         Nastav cas na aktivacni cas acttime v AZ
         Proved popis chovani udalosti event v AZ
}
cas = T_END; konec simulace
\end{lstlisting}

\section{Architektura simulačního modelu/simulátoru}

Prvky simulátoru jsou implementovány v pěti hlavních třídách (\texttt{calendar.cpp}, \texttt{facility.cpp} ...). Tyto třídy jsou dále využívány pro simulaci, kterou najdeme v \texttt{simulation.cpp}? Níže jsou jednotlivé třídy popsány.

\subsection{Kalendář událostí}
\subsection{Fronta}
\subsection{Obslužné linky}
\subsection{Zařízení} 

\section{Podstata simulačních experimentů a jejich průběh}
\section{Shrnutí simulačních experimentů a závěr}

\newpage

\bibliographystyle{myplain}
\bibliography{bibliography}

%\section*{Použitá literatura} \label{literatura}
%\begin{itemize}
%\item [1] Karen R. Sollins, “The TFTP Protocol” IEN 133, 29.1.1980. [Online]. Dostupné z %\url{http://www.postel.org/ien/txt/ien133.txt}
%\item [2] Karen R. Sollins, “The TFTP Protocol (revision 2)”RFC1350, 29.7.1992. [Online]. Dostupné z %\url{http://tools.ietf.org/html/rfc1350}
%\item [3] G. Malkin, Bay Networks, A. Harkin, “TFTP Option Extension” RFC2347, 1998. [Online]. Dostupné z %\url{http://tools.ietf.org/html/rfc2347}
%\item [4] G. Malkin, Bay Networks, A. Harkin, “TFTP Blocksize Option” RFC2348, 1998. [Online]. Dostupné z %\url{http://tools.ietf.org/html/rfc2348}
%\item [5] G. Malkin, Bay Networks, A. Harkin, “TFTP Timeout Interval and Transfer Size Options” RFC2349, 1998. [Online]. %Dostupné z \url{http://tools.ietf.org/html/rfc2349}
%\end{itemize}

\end{document}